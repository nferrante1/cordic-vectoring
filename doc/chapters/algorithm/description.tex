\section{Description}
CORDIC (COordinate Rotation Digital Computer), known also as Volder's algorithm
is a simple and hardware-efficient iterative algorithm used to compute a wide
range of elementary, trigonometric and hyperbolic functions. \newline
It was conceived by Jack E. Volder in 1956 as an accurate and performant
real-time digital replacement for B-58 Bomber's analog resolver.\newline
CORDIC idea is to translate complex operations in vectors' rotations that can be
efficiently implemented in hardware as shift and add operations.

\subsection{Mathematical Background}
Vector's rotation is the basis of CORDIC algorithm. Let's suppose to have an
input vector $(x_0, y_0)$ and that we want to rotate it of an arbitrary angle 
$\theta$. Suppose that we choose the origin as the center of the rotation, so we
can compute the rotated vector as: \newline
$\left.\begin{gathered}
	x_1 &= x_0cos\theta - y_0sin\theta \\
	y_1 &= x_0sin\theta + y_0cos\theta 
	\end{gathered}
\right\}$
